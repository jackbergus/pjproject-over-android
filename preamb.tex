\documentclass[12pt,italian]{amsbook}
\usepackage{framed}
\usepackage[T1]{fontenc}
\usepackage[utf8]{inputenc}
\usepackage[utf8]{inputenx}
\usepackage[a4paper]{geometry}
%\geometry{verbose,tmargin=4cm,bmargin=4cm,lmargin=1.25cm,rmargin=1.25cm,headheight=2cm,headsep=2cm,footskip=2cm}



%% TOCs
\setcounter{secnumdepth}{9}
\setcounter{tocdepth}{9}
\usepackage[tight,italian]{minitoc}
\usepackage[italian]{varioref}
\usepackage[unicode=true, pdfusetitle,
 bookmarks=true,bookmarksnumbered=false,bookmarksopen=false,
 breaklinks=true,pdfborder={0 0 1},backref=false,colorlinks=false]
 {hyperref}


%% GRAPHICS
\usepackage[all]{xy}
\usepackage{color}
\usepackage{graphicx}
\usepackage{booktabs}
\usepackage{tabularx}
\usepackage{rotating}
\usepackage{subfig}
\usepackage{ulem}
%% NOTA: makeidx già supportato da ams



%% MATH
\usepackage{paralist}
\usepackage{braket}
\usepackage{array}
\usepackage{float}
\usepackage{units}
\usepackage{textcomp}
\usepackage{mathpazo}
\usepackage{amsfonts}
\usepackage{amsmath,amsthm}
\usepackage{amsthm}
\usepackage{mathpazo}
\usepackage{amssymb}
\usepackage{mathdots}
\usepackage{synttree}
\usepackage{mathabx}

%QUOTES
\usepackage{quoting}
\usepackage{epigraph}



%%%%%%%%%%%%%%%%%%%%%%%%%%%%%%%%%%%%%%%%%%%%%%%%%%%%%%%%%%%%%%%%%%% BIBLIOGRAPHY
\usepackage[italian]{babel}
\usepackage[babel]{csquotes}
\usepackage[style=alphabetic,backref,hyperref,babel=hyphen]{biblatex}
\bibliography{Biblio}

\DeclareBibliographyCategory{Bibliografia}
\DeclareBibliographyCategory{Sitografia}	

\addtocategory{Bibliografia}{tesi:binder, tesi:nexus, libro:games, libro:embedded, man:gapil, art:middleware, libro:jni, libro:carli, art:apex}
\addtocategory{Sitografia}{man:pjsip,slide:aporting,site:jollen,art:notdroid,site:marakAndroidInternals,site:marakRemixing, slide:androidhfeat, site:anonBinder, slide:zhIPC}

\defbibheading{Bibliografia}{\section*{Bibliografia}}
\defbibheading{Sitografia}{\section*{Sitografia}}

\makeatletter

%%%%%%%%%%%%%%%%%%%%%%%%%%%%%% LyX specific LaTeX commands.
\newcommand{\noun}[1]{\textsc{#1}}
\let\SF@@footnote\footnote
\def\footnote{\ifx\protect\@typeset@protect
    \expandafter\SF@@footnote
  \else
    \expandafter\SF@gobble@opt
  \fi
}
\expandafter\def\csname SF@gobble@opt \endcsname{\@ifnextchar[%]
  \SF@gobble@twobracket
  \@gobble
}
\edef\SF@gobble@opt{\noexpand\protect
  \expandafter\noexpand\csname SF@gobble@opt \endcsname}
\def\SF@gobble@twobracket[#1]#2{}
%% Because html converters don't know tabularnewline
\providecommand{\tabularnewline}{\\}
\floatstyle{ruled}


%%%%%%%%%%%%%%%%%%%%%%%%%%%%%% Textclass specific LaTeX commands.
\numberwithin{section}{chapter}
\numberwithin{equation}{section}
\numberwithin{figure}{section}
 \newtheorem*{defn*}{Definizione}
 \newtheorem{defin}{Definizione}
  \newtheorem*{prop*}{Proposizione}
  \newtheorem*{problem*}{Problema}
  \newtheorem{problem}{Problema}[section]
  \newtheorem*{example*}{Esempio}
  \newtheorem{example}{Esempio}[chapter]
  \newtheorem{ex}{Esercizio}[chapter]
  \newtheorem*{lem*}{Lemma}
  \newtheorem*{thm*}{Teorema}
\newtheorem{thm}{Teorema}[section]
\newtheorem{coroll}{Corollario}[section]
\newtheorem{theorem}{Teorema}[section]
  \newtheorem{prop}{Proposizione}[section]
  \newtheorem{lem}{Lemma}[section]
  \newtheorem{defn}{Definizione}[section]
  

%%%%%%%%%%%%%%%%%%%%%%%%%%%%%% LSTLISTINGs Settings.
\usepackage{listingsutf8}

%TEXFONTS
\usepackage[scaled]{beramono}

%%FLOATING LATEXs
\newfloat{algorithm}{tbp}{loa}
\floatname{algorithm}{Listato}

  \definecolor{lightgray}{gray}{0.95}
\DeclareCaptionFont{white}{\color{white}}
\DeclareCaptionFont{lightgray}{\color{lightgray}}

 \lstset{
         basicstyle=\footnotesize\ttfamily, % Standardschrift
         %numbers=left,               % Ort der Zeilennummern
         numberstyle=\tiny,          % Stil der Zeilennummern
         %stepnumber=2,               % Abstand zwischen den Zeilennummern
         extendedchars=true,         %
         keywordstyle=\bfseries,
    	 frame=single,
         stringstyle=\color{red}\ttfamily, % Farbe der String
         showspaces=false,           % Leerzeichen anzeigen ?
         showtabs=false,             % Tabs anzeigen ?
         inputencoding=utf8/latin1,
         %backgroundcolor=\color{lightgray},
         showstringspaces=false,      % Leerzeichen in Strings anzeigen ?  
         frameround=fttt,
         backgroundcolor=\color{lightgray},
         xrightmargin=-13pt,
         breaklines=true
 }
 \lstloadlanguages{% Check Dokumentation for further languages ...
         %[Visual]Basic
         %Pascal
         %C
         %C++
         %XML
         %HTML
         Java
 }
    %\DeclareCaptionFont{blue}{\color{blue}} 

  %\captionsetup[lstlisting]{singlelinecheck=false, labelfont={blue}, textfont={blue}}
  \usepackage{caption}

\DeclareCaptionFormat{listing}{\colorbox[cmyk]{0.43, 0.35, 0.35,0.01}{\parbox{\textwidth}{\hspace{15pt}#1#2#3}}}
\captionsetup[lstlisting]{format=listing,labelfont=white,textfont=white, singlelinecheck=false, margin=0pt, font={bf,footnotesize}}

%%Visual Basic
\lstset{defaultdialect=[Visual]Basic}

%%OCAML
\lstnewenvironment{ocaml}[1][]{\lstset{language=[Objective]Caml}}{}


%%CONCURRENT G
\lstdefinelanguage{concurrentg}  {
morekeywords={while, cobegin, coend, do, if, true, false, else, void, int, new, for, shared, let, rec, then, to, match, with, raise, done, begin, end, not, each, in, return},
morecomment=[s][]{/*}{*/},
morestring=[b][\color{red}]"
}
\lstnewenvironment{concurrentg}[1][]{\lstset{language=concurrentg,#1}}{}



%% SCHEME
\lstdefinelanguage{Scheme}  {
morekeywords={set,first,cons,make,struct,cond,true,false},
morecomment=[l]{;},
}
\lstnewenvironment{scheme}[1][]{\lstset{language=Scheme,#1}}{}


\lstnewenvironment{pascal}[1][]{\lstset{language=Pascal,#1}}{}
\lstnewenvironment{clang}[1][]{\lstset{language=C,#1}}{}

\lstnewenvironment{bash}[1][]{\lstset{language=bash,#1}}{}

\lstnewenvironment{xml}[1][]{\lstset{language=XML,#1}}{}
\lstnewenvironment{java}[1][]{\lstset{language=Java,#1}}{}
\lstnewenvironment{cpp}[1][]{\lstset{language=C++,#1}}{}


%%PSEUDOCODE
\lstnewenvironment{pseudolus}[1][]{\lstset{
morekeywords={For, Each, if, then, in, End, for, to, do, var, else},
frame=shadowbox,
#1
}}{}
\newenvironment{pseudolang}{\begin{center}\begin{pseudolus}}{\end{pseudolus}\end{center}}




%%CENTER LINE CODE
\newcommand{\shcode}[1]{\begin{center}\texttt{#1}\end{center}}


%%%%%%%%%%%%%%%%%%%%%%%%%%%%%%%%%%%%%%%%%%%%%%%%%%%%%%%%%%%%%%%%%%%%%%%%%%%%%%%%

% mie definizoni 
\newcommand{\diam}{\item[$\diamond$]} % punto elenco diamante
\newcommand{\cerc}{\item[$\circ$]} % punto elenco cerchio
\newcommand{\suspence}{[\dots]} % ellissi
\newcommand{\opquote}{«}
\newcommand{\clquote}{»}
\newcommand{\reffig}{Figura}
\newcommand{\singlediam}[1]{\begin{itemize}\diam #1\end{itemize}} % un solo punto elenco con diamante
\newcommand{\ttilde}{\textasciitilde{}  } % scrittura semplificata della tilde
\renewcommand{\*}{\cdot}
\renewcommand{\d}{\$}
\renewcommand{\emph}{\textit}
\newcommand{\myquote}[2]{«\emph{#1}»(#2)}
\newcommand{\myquotu}[1]{«\emph{#1}»}
\newcommand{\textbb}[1]{$\mathbb{#1}$}

\newcommand{\nat}{\mathbb N}
\newcommand{\real}{\mathbb R}
\newcommand{\bool}{\mathbb B}
\newcommand{\lang}{\mathcal L}
\newcommand{\down}{\downarrow}
\newcommand{\up}{\uparrow}
\newcommand{\IIF}{\Leftrightarrow}
\newcommand{\iif}{\leftrightarrow}


%% Form Net
\newsavebox{\sembox}
\newlength{\semwidth}
\newlength{\boxwidth}

\newcommand{\Sem}[1]{%
\sbox{\sembox}{\ensuremath{#1}}%
\settowidth{\semwidth}{\usebox{\sembox}}%
\sbox{\sembox}{\ensuremath{\left[\usebox{\sembox}\right]}}%
\settowidth{\boxwidth}{\usebox{\sembox}}%
\addtolength{\boxwidth}{-\semwidth}%
\left[\hspace{-0.3\boxwidth}%
\usebox{\sembox}%
\hspace{-0.3\boxwidth}\right]%
}



 % definizione dell'elenco con le lettere (wikipedia.en)
%\arabic 	1, 2, 3 ...
%\alph 	a, b, c ...
%\Alph 	A, B, C ...
%\roman 	i, ii, iii ...
%\Roman 	I, II, III ...
%\fnsymbol 	Aimed at footnotes (see below), but prints a sequence of symbols.
 % come consigliato dalla Nasa (http://www.giss.nasa.gov/tools/latex/ltx-222.html):
 % \renewcommand{\labelenumi}{\emph{(\alph{enumi})}}
\newenvironment{itmletters}[1]{\begin{enumerate}\renewcommand{\labelenumi}{\emph{#1}}}
                              {\end{enumerate}}
\newenvironment{stab}{\begin{table}\begin{shaded}}
                              {\end{shaded}\end{table}}
 % questo comando crea l'elenco con i letterali definiti sopra
\newcommand{\atm}{\alph{enumi}}
 % questo comando serve per indicare in quello sopra l'enumerazione della prima serie di numeri
 % come di 
\newcommand{\itm}{\roman{enumi}}
\newcommand{\cent}[1]{\begin{center}#1\end{center}}
\newcommand{\textcal}[1]{$\mathcal{#1}$}
                        
\newcommand{\startfrom}[1]{\setcounter{enumi}{#1}}
\newcommand{\ttitem}[1]{\item \texttt{#1}}
\newcommand{\tleq}{$\leq$} 
\newcommand{\tgeq}{$\geq$}
\newcommand{\lng}{\mathcal L}
\newcommand{\Jj}{$\drsh$}

\newcommand{\st}[1]{\tt{\small{ #1}}}

\newtheorem{prob}{Problema}[section]
\newtheorem{xmpl}{Esempio}[section]
\newtheorem{propr}{Proprietà}[section]


\newcommand{\AOSP}{\texttt{\small \$AOSP}}
\newcommand{\PJA}{\texttt{\small \$PJA}}
\long\def\syfoot[#1]#2{\begingroup%
\def\thefootnote{\fnsymbol{footnote}}\footnote[#1]{#2}\endgroup} 

%%% problema: questa tecnica tuttavia scalza le eventuali immagini presenti
%%% all'interno del documento corrente.
\newcommand{\teorema}[3]{

\medskip

	\fbox{\parbox{1\linewidth}{
	\begin{#1} #2 \end{#1}
	\begin{proof} #3 \end{proof}
	}}
	
\medskip

}
\newcommand{\tteorema}[4]{
\medskip


	\fbox{\parbox{1\linewidth}{
	\begin{#1}[#2] #3 \end{#1}
	\begin{proof} #4 \end{proof}
	}}
\medskip


}

\hypersetup{
 pdfsubject={Progetto},
 pdfkeywords={informatica}}

\renewcommand\lstlistingname{Listato }
\renewcommand\lstlistlistingname{Listati}

\makeindex
\makeatother
\dominitoc
