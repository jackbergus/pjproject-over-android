\title{Analisi sul porting Android di PJSIP}
\author{Giacomo Bergami}
%\maketitle

\begin{list}{}{
  \setlength{\topsep}{0pt}
  \setlength{\leftmargin}{0pt}%
  \setlength{\rightmargin}{0pt}%
  \setlength{\listparindent}{0pt}%
  \setlength{\itemindent}{0pt}%
  \setlength{\parsep}{0pt}%
 }%
\item[]
\thispagestyle{empty}
\begin{center}
\item[]
{{\Large{\textsc{Alma Mater Studiorum $\cdot$ Universit\`a di
Bologna}}}} \rule[0.1cm]{15.8cm}{0.1mm}
\rule[0.5cm]{15.8cm}{0.6mm}
{\small{\bf FACOLT\`A DI SCIENZE MATEMATICHE, FISICHE E NATURALI\\
Corso di Laurea Triennale in Informatica}}
\end{center}
\vspace{15mm}
\begin{center}
{\LARGE{\bf Pjproject su Android:}}\\
\vspace{5mm}
{\LARGE{\bf uno scontro su più livelli}}\\
\vspace{19mm} {\large{\bf Tesi di Laurea in Architettura degli Elaboratori}}
\end{center}
\vspace{40mm}
\par
\noindent
\begin{minipage}[t]{0.47\textwidth}
{\large{\bf Relatore:\\
Chiar.mo Prof.\\
Ghini Vittorio}}
\end{minipage}
\hfill
\begin{minipage}[t]{0.47\textwidth}\raggedleft
{\large{\bf Presentata da:\\
Bergami Giacomo}}
\end{minipage}
\vspace{20mm}
\begin{center}
{\large{\bf Sessione II\\%inserire il numero della sessione in cui ci si laurea
Anno Accademico 2011-2012}}%inserire l'anno accademico a cui si è iscritti
\end{center}
\end{list}
\pagebreak

\tableofcontents


\begin{list}{}{
  \setlength{\topsep}{\textwidth $ $}
  \setlength{\leftmargin}{0pt}%
  \setlength{\rightmargin}{0pt}%
  \setlength{\listparindent}{0pt}%
  \setlength{\itemindent}{0pt}%
  \setlength{\parsep}{0pt}%
 }%
\item[]
\thispagestyle{empty}
\epigraph{Non senza fatiga si giunge al fin.}%
{\textit{G. Frescobaldi}\\''Toccata Nona`` dalle «Toccate e partite d'intavolatura di cimbalo, Libro Secondo»}
\end{list}
\pagebreak

\pagebreak
