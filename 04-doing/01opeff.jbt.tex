\chapter{Tentativi di porting e considerazioni effettuate}
\minitoc\mtcskip

Esporrò di seguito per punti le procedure effettuate allo scopo di realizzare il
porting dell'applicazione, ed a quali conclusioni sia giunto ogni volta. 
Elenco di seguito ulteriori considerazioni effettuate, non descritte all'interno
di questo capitolo ma presenti all'interno della Tesi:
\begin{itemize}
\diam Interazione tra Emulatore e Macchina Ospite: v. Sottosezione \vref{subsec:commtwoemu}.
\diam Tentativi di configurazione per la crosscompilazione: v. Sezione \vref{sec:tentconfcross}.
\diam Analisi del sorgente e modifica nel supporto dei file WAVE: v. Sezione \vref{sec:consWAVE}.
\diam Modifica nel sorgente dell'AOSP Source \vref{sec:modifaosp}.
\end{itemize}

\section{Considerazioni sulla crosscompilazione}\label{sec:conscrossocmp}
Una breve descrizione dei tentativi effettuati per la crosscompilazione
è fornita nell'Appendice alla Sezione \vref{sec:tentconfcross}, tramite il
particolareggiamento sul cambiamento delle configurazioni.

Proseguo ora nella descrizione dei tentativi di porting:
\begin{enumerate}
\item Come prima cosa, ho provato a compilare \texttt{\small pjsua} specificando unicamente
       in quale percorso trovare il binario di cross-compilazione, effettuando
       l'export dell'unica variabile CC. 
       
       Proprio a questo credo sia dovuto il primo errore di compilazione: di
       fatti lo stack del \texttt{pc} deducibile dal \texttt{logcat} forniva come
       ultimi binari \texttt{/system/bin/linker} e \texttt{/system/lib/libc.so}.
       
       Questo mostrava che la procedura di linking dinamica funzionava a dovere,
       che effettivamente il linker aveva utilizzato la libreria richiesta,
       ma l'esecuzione di \texttt{\_\_set\_errno} era errata. Con il senno di poi,
       e conoscendo l'istruzione \texttt{nm} per andare ad indagare i simboli
       contenuti all'interno dei file binari, ho visto che \texttt{\_\_set\_syscall\_errno}
       era posizionata proprio all'inizio del file, e con ogni probabilità è
       stata scelta come funzione \textit{entry point}. Non era stato quindi compilato
       correttamente per l'architettura ARM.
\item In seguito ho riscontrato lo stesso errore anche compilando la libreria 
       \textit{pjlib} nel branch di Android, in quanto (con ogni probabilità) credevo
       che il problema consistesse nel sorgente, e non nel metodo da me
       adottato per effettuare cross compiling.
\item Poi ho trovato i flag di compilazione per architettura ARMv5, ed
       ho provato ad applicare quelli allo scopo di effettuare cross compilation.
       Ho provveduto quindi a modificare direttamente  i flag di linking e compilazione
       in \texttt{\small configure-android},
       senza ancora utilizzare il file .mak
       
       Con ogni probabilità anche l'esecuzione della syscall con i seguenti parametri:
       \begin{center}
       \texttt{\small sigaction(48472, {0xb00144c4, [], SA\_RESTART}, {0xb00144c4, [], SA\_RESTART}, 0) = -1 EINVAL}
       \end{center}
       (che portava ad un SIGFAULT in quanto non era definito da nessuna parte un
       segnale con codice 48472) indicava un errato modo di effettuare cross-compilazione,
       o per i flag che indicano le caratteristiche dell'EABI del processore, o
       ancora per una non corretta endianess riconosciuta in fase di configurazione.
\item Osservando che queste syscall non vengono eseguite correttamente,
       suppongo che questo sia dovuto ad una carenza di permessi dell'utente
       che esegue le istruzioni. 
       
       Come ho avuto modo di dimostrare precedentemente, ritengo col senno di
       poi che questo fosse imputabile ad una non corretta
       cross-compilazione, anche se il problema dei permessi sarebbe  
       comunque prima o poi emerso.
       
       Ho osservato, leggendo la documentazione di Google, che 
       ogni applicazione viene eseguita all'interno di una ``sandbox'', per 
       uscire parzialmente dalla quale è necessario esibire al sistema la 
       richiesta di speciali permessi di esecuzione. Avendo prima d'allora
       effettuato programmazione Android unicamente in Java, ho fornito tali
       permessi unicamente tramite un file, detto ``Manifest'', all'interno
       del quale esplicitavo la necessità di connettermi alla rete Internet, piuttosto
       di scrivere all'interno del filesystem. Da quanto mi risultava inoltre da
       procedure di rooting, con i permessi di superutente si sarebbe potuto
       uscire dalla sandbox. Questo è peraltro confermato da informazioni che
       ho acquisito successivamente permesso lato \textit{service} Java.
\item In quanto ritengo ``unsafe'' in termini di garanzia del dispositivo reale
       effettuare rooting sul mio primo dispositivo (ovvero l'Olivetti Olitab, 
       che tra l'altro è scarsamente supportato
       in ambito cracking rispetto ad altri più mainstream), effettuo una prima
       procedura di rooting sull'emulatore; questa è già stata discussa nella
       Sottosezione \vref{subsec:rootingemu}. 
       
       A riprova del fatto che la
       causa del mancato funzionamento del binario non era da addebitare al mancato rooting del dispositivo, 
       neanche in questo modo nell'emulatore
       si risolve l'\texttt{EINVAL} di cui sopra. A questo punto, imputando 
       il problema ad una limitazione effettiva di tale dispositivo (cosa
       plausibile visti i miei trascorsi con la programmazione Java), mi decido
       ad effettuare la procedura di rooting anche sul mio dispositivo personale.
       
       Ad ulteriore riprova di ciò che sto più volte ribadendo,
       anche utilizzando il tablet l'applicazione non sembra partire. Inoltre, non 
       essendo implementato in questo dispositivo il binario \texttt{\small strace}\footnote{
       Provai ad utilizzare un binario trovato in Rete: tuttavia ancora adesso,
       su qualsiasi binario tracciato (compreso l'\texttt{ls} effettivamente 
       disponibile all'interno del dispositivo personale in questione). effettua
       unicamente il tracciamento della syscall \texttt{exec*}.}, ciò rende 
       difficoltosa l'interpretazione dei motivi dell'errore.
       
       Effettuo quindi la comparazione tra gli output del LogCat (più verboso ma
       meno informativo della \texttt{\small strace}) dell'emulatore e quello del dispositivo 
       personale in questione. Noto solo così per la prima volta la discrepanza
       tra i due output. Inizio a pensare per la prima volta che l'errore di esecuzione
       sia dovuto ad una disparità tra la costruzione architetturale ed i flag
       di compilazione.
\item Riguardando in una fase successiva la definizione di \texttt{\small SIGILL}, noto che
       questo segnale è effettivamente lanciato al processo che tenta di eseguire
       un'istruzione malformata, sconosciuta o che richiede particolari privilegi.
       
       Esclusa  l'ipotesi della necessità di acquisire particolari permessi,
       in quanto abbiamo supposto sin dall'inizio che la procedura di rooting
       garantisca l'accesso al superuser e quindi potenzialmente a tutte le istruzioni
       fornite dal kernel (o almeno alla maggior parte), è evidente che l'istruzione
       debba essere malformata, e conseguentemente sconosciuta, proprio a causa
       di una differente tipologia tra hardware reale e hardware di compilazione.
       L'errore non era localizzabile nella libreria in sé, in quanto caricata 
       dinamicamente, ma nell'istruzione richiesta dal binario da me compilato.
       
       A questo punto arrivo a concludere che il problema possa essere insito 
       nei flag di crosscompilazione: da ciò segue che le direttive al compilatore
       fornite dal branch di Android per effettuare la crosscompilazione, al
       quale mi appoggiavo, non erano sufficienti al fine di effettuare una corretta
       crosscompilazione. Di fatti, dispositivi differenti dovrebbero necessitare
       di differenti configurazioni in fase di compilazione.
\item In riferimento al tracciamento dei processi figli, che sembrano non essere
	tracciabili da strace in Android, ritengo opportuno spostare la fase
	di testing su applicazioni che non prevedano inizializzazioni o creazioni
	di thread come \textit{\small pjsua}, almeno per una fase iniziale, visto l'esiguo
	numero di informazioni che sono ottenibili dal LogCat (Non viene di
	fatti indicato quale syscall è stata chiamata, ma solo l'avanzare del \textit{\small pc}
	nel corso dell'esecuzione, di per se non molto significativo, se non a patto
	di ricostruirlo analizzando il binario generato). Proseguo quindi il testing
	su \texttt{\small test-pjsip}.
	
	Documentandomi meglio, tendo poi 
	ad escludere che Android non permetta la creazione
	di processi figli, in quanto la libreria \textit{pthread} è 
	nativamente supportata all'interno di \textit{Bionic}.
\item Riporto le osservazioni che avevo effettuato sull'esecuzione di \texttt{\small test-pjsip}:
	l'esecuzione dell'istruzione \texttt{\small usage} al posto di \texttt{\small main}
	era proprio da imputare alla mancanza dell'entry-point \texttt{\small\_start}.
	Non conscio del fatto che questo fosse già implementato all'interno di
	\texttt{\small crtbegin}, in quanto convinto che questo dovesse comunque essere 
	implementato all'interno della \texttt{\small libc}, decido di utilizzare l'implementazione
	Assembly di tale entry point, documentandomi su quale fosse la procedura
	di inizializzazione di libc, e provando ad effettuare il disassembly
	di alcuni binari di prova per Android. Trovo poi conferma di questa
	procedura documentandomi in rete, ed appoggiandomi a quel risultato 
	maggiormente raffinato rispetto a quello prodotto da me inizialmente.
	
	Solo in seguito alla redazione di una prima bozza di Tesi ed alla
	visione dei sorgenti di \textit{Bionic}, scopro che tale implementazione era contenuta all'intero
	di \texttt{crtbegin}, unicamente tramite la somiglianza tra il sorgente da
	me prodotto, e quello fornito da Google. In seguito questa supposizione
	trae conferma dall'utilizzo di \texttt{nm} sull'oggetto binario in questione.





\item Avevo precedentemente imputato l'ulteriore interruzione dell'esecuzione
	di \texttt{\small pjsua} all'errata esecuzione della syscall uname: tuttavia, a riprova
	che questa è effettivamente disponibile, il file da me scritto in C e
	crosscompilato allo scopo di ottenere la versione correntemente 
	utilizzata nel mio dispositivo, ha ribadito (come tra l'altro evidenziato
	da un'analisi più accorta dell'output della strace) come questa sia
	effettivamente accessibile.
	
	Come evidenziato, il problema dovrebbe essere costituito dall'accesso
	all'indirizzo NULL di memoria (Compare di fatti da Logcat un ``Segmentation Fault 
	at 0x0''). Ciò tuttavia mi lascia alquanto perplesso, in quanto pjsua su adm64
	non ha mai causato quel genere di problemi, ma veniva eseguito correttamente.
	Da qui nasce appunto l'esigenza di proseguire il testing su di un
	sorgente meno ``intricato'' dal punto di vista del debugging.
\item Come tra l'altro appena ribadito, il mio dubbio
	risiede nel fatto che, una volta liberato dai 
	vari permessi, l'applicazione sembra ancora non proseguire correttamente
	nell'esecuzione. In particolare, dopo l'ultima stampa a video, 
	noto un palese delay tra questa e la interruzione dell'operazione.
	
\item Mi riservo un'ultima considerazione di mero carattere speculativo in
	merito all'architettura Android, e non correlata con il sorgente specifico
	se non per lo scopo che mi prefiggo, ovvero quello di effettuare il porting
	dell'applicazione per Android, circa la lettura di ``Embedded Android''
	di Karim Yaghmour.
	
	L'autore tuttavia sostiene che le applicazioni ``native'' non dovrebbero
	essere soggette a restrizioni, anche se poi asserisce che ``tutto ciò che
	avviene di interessante è gestito da 50 servizi (tra cui cito \texttt{\small
	netstat} e \texttt{\small permission}) di sistema, dove l'interazione è gestita da un Binder
	(\texttt{\small /dev/binder}).
	
	Inizialmente ritengo che siano necessarie tecniche di ``hookup''\footnote{I meccanismi più diffusi tramite i quali 
       si effettua lo hacking del dispositivo sono il ``Privilege Escalation
       Attack'' e il cosiddetto ``hooking'', anche se la procedura di rooting
       varia in base alla versione di Android ed alle caratteristiche del 
       dispositivo.} per 
	evitare che le \textit{system call} vengano intercettate dal sistema operativo:
	tuttavia a questo punto non ero ancora consapevole dei meccanismi di IPC
	che collegano tramite \textit{upsyscall} i binari ai servizi di sistema, in
	quanto confinavo il Binder ad un ambito strettamente legato alla DVM ed
	a Java.
\item Seguono i tentativi di configurazione degli script di compilazione 
	descritti nella Sezione \vref{sec:tentconfcross}.
\item Dopo un'attenta analisi dei warning in fase di compilazione, mi accorgo
	di come rimangano ulteriori messaggi segnalanti la mancata definizione di
	\texttt{\small \_start}. Ciò era  dovuto all'indicazione del binario 
	\texttt{\small crt0.o} unicamente all'interno della variabile \texttt{\small PJ\_LDLIBS},
	che credevo  contenuta all'interno di tutte le variabili per il 
	linking.
	In seguito alle configurazioni definitive già mostrate
	nel Capitolo \vref{cha:pjproject}, ottengo per la prima volta l'esecuzione
	completa di \texttt{\small pjsua}.
\end{enumerate}



