\section{Considerazioni sulla riproduzione dei file WAVE}\label{sec:consWAVE}
\begin{enumerate}
\item Come descritto nella Sottosezione \vref{subsec:impossibemuandroid}, era
	impossibile da emulatore controllare l'effettiva riproduzione dei files,
	in quanto questo non supporta l'emulazione delle periferiche audio.
	Dopo aver notato, prima sull'Olipad e poi anche sul Galaxy Nexus, che
	l'esecuzione contemporanea di due istanze di \texttt{\small pjsua} 
	provoca dei problemi nell'esecuzione del programma, in questa prima
	fase mi concentro sul testing dei due dispositivi, utilizzandone uno
	per la riproduzione dell'audio, e l'altro per la trasmissione dello stesso
	all'interno del canale di comunicazione. 
	
	Questi due dispositivi potranno interagire grazie all'utilizzo di un 
	\textit{router} domestico, che fornirà loro gli indirizzi IP per poter 
	comunicare all'interno della rete locale.
	
	Per l'analisi del problema relativo all'esecuzione contemporanea di 
	due istanza di \texttt{\small pjsua}, si veda la Sezione \vref{sec:modifaosp}.
	
\item Prima di effettuare l'analisi approfondita dell'errore, provo ad attivare
	o disattivare alcuni codec in fase di compilazione. Nel caso specifico,
	come primo tentativo aggiungo i seguenti:
	\begin{bash}
--enable-ext-sound --disable-speex-codec --disable-speex-aec 
	\end{bash}
	Effettuo questa scelta in quanto l'applicazione mostra, con la configurazione
	di default, di utilizzare il codec \texttt{\small speex}:
	\begin{bash}
>15:24:37.407  pjsua_media.c  ......Audio updated, stream #0: speex (sendrecv)
	\end{bash}
	Una volta ricompilato il sorgente, provo ad eseguire un'istanza
	di \texttt{\small pjsua} come registratore sul Tablet Olipad, mentre 
	scelgo di far riprodurre la traccia audio all'interno del Galaxy Nexus.
	In questo caso riscontro quindi un problema con la deallocazione del
	codec GSM, ed in particolare l'applicazione termina lato riproduttore
	con un Segmentation Fault. In particolare il LogCat palesa l'errore
	all'interno della funzione  \texttt{\small gsm\_dealloc\_codec}.
	
	In particolare noto che, dopo una prima esecuzione del programma con
	errore, facendolo ripartire in un secondo momento sulla stessa porta,
	si riscontra un errore all'atto del \textit{binding} della porta, come
	mostrato dallo stesso output dell'applicazione:
	\begin{bash}
22:57:24.663   pjsua_core.c  bind() error: Address already in use [status=120098]
	\end{bash}
	
	Per evitare l'errore nella gestione di \texttt{\small gsm\_dealloc\_codec},
	provo a disabilitare anche quella libreria in fase di configurazione con 
	\texttt{\small --disable-gsm-codec}. Provando quindi ad eseguire nuovamente
	il binario sui due dispositivi, come già illustrato sopra: ottengo ora
	ancora una volta nel dispositivo al quale è preposta la registrazione
	l'errore causato dall'asserzione già osservato, ovvero:
        \begin{bash} 
assertion "cport->rx_buf_count <= cport->rx_buf_cap" failed: file "../src/pjmedia/conference.c", line 1498, function "read_port"
\end{bash}

	Ciò evidenzia come il problema sia indipendente dal codec utilizzato,
	in quanto è comune sia alla prima configurazione e sia all'ultima.
	
\item Proseguendo ora nell'analisi dell'errore, effettuo i seguenti controlli
	nel codice sorgente:
	\begin{itemize}
	\item Controllo che effettivamente si ottengano le informazioni corrette
		dal file: questo è verificato dal fatto che si 
		ottengono le informazioni dallo header del file WAVE.
	\item Controllo se esista  un formato audio riproducibile
		e che non causi errori. Mentre utilizzando il file
		\url{http://www.nch.com.au/acm/11k16bitpcm.wav} si continua
		a verificare il problema di asserzione, così come tutti gli altri
		files ottenibili dal sito, non riscontro alcun problema con 
		il file d'esempio:
		\begin{center}
		\texttt{\small \PJA/tests/pjsua/wavs/input.8.wav}
		\end{center}	
		Adduco a motivazione il fatto che, il file correntemente aperto,
		abbia le stesse caratteristiche del bridge di comunicazione, e 
		che per questo si ottenga il bypass del controllo di asserzione
		del controllo di cui sopra, in quanto in quel caso i dati vengono
		scritti direttamente all'interno della porta del bridge.
	\end{itemize}
\item Per verificare questa mia ultima supposizione, aggiungo delle stampe di 
	controllo all'interno della funzione \texttt{\small read\_port}: da queste
	scopro che anche nel caso del file ottenibili dai test del sorgente si 
	entra nella gestione dei files con configurazione differente dal bridge
	di comunicazione. 
	
	Posso inoltre controllare come, in questo caso, avvenga effettivamente
	la conversione stereo/mono o mono/stereo ed il resampling audio in base
	al differente sample rate, copiando poi le informazioni all'interno
	del buffer RX per l'uscita delle informazioni verso il bridge di 
	comunicazione. In particolare l'output dell'esecuzione di \texttt{\small pjsua} 
	file d'esempio corrente è il seguente:
\begin{bash}
21:21:49.299   conference.c  bufcount = 160, bufcap = 160, tmpsize=320, spf=160
21:21:49.308   conference.c  WARNING: EXCEEDING. bufcount = 0, bufcap = 160, tmpsize=320, spf=160
21:21:49.308   conference.c  bufcount = 160, bufcap = 160, tmpsize=320, spf=160
\end{bash}
	Mentre l'esecuzione di un file con caratteristiche audio differenti è il 
	seguente:
\begin{bash}
21:19:09.101   conference.c !WARNING: EXCEEDING. bufcount = 0, bufcap = 429, tmpsize=438, spf=219
21:19:09.102   conference.c  bufcount = 219, bufcap = 429, tmpsize=438, spf=219
21:19:09.102   conference.c  WARNING: EXCEEDING. bufcount = 219, bufcap = 429, tmpsize=438, spf=219
21:19:09.102   conference.c  bufcount = 438, bufcap = 429, tmpsize=438, spf=219
assertion "cport->rx_buf_count <= cport->rx_buf_cap" failed: file "../src/pjmedia/conference.c", line 1513, function "read_port"
\end{bash}	
	dove $tmpsize$ indica la dimensione del $frame$ costituito da più $samples$,
	ovvero:
	\begin{center}
	\texttt{\small tmpsize = cport->samples\_per\_frame * cport->bytes\_per\_sample}
	\end{center}
\item 	In particolare la spiegazione della scelta delle dimensioni del buffer
	è confermata dal sorgente:
	\begin{center}
	\texttt{\small \PJA/pjmedia/src/pjmedia/resample\_resample.c}
	\end{center}
	si vuole tenere la prima parte del buffer allo scopo di memorizzare le
	informazioni appena metabolizzate, in attesa che arrivino frame sufficienti
	per effettuare l'operazione di resampling. Questo tuttavia presuppone
	una gestione errata della stessa area di memoria da parte degli sviluppatori 
	e, in particolare, si vede che l'asserzione è dovuta al fatto che la 
	dimensione del buffer (\texttt{\small bufcount}, ovvero \texttt{\small cport->rx\_buf\_count})
	è nettamente superiore alla dimensione della sua capacità massima
	(\texttt{\small bufcap}, ovvero \texttt{\small cport->rx\_buf\_cap}), mentre
	nel caso della gestione corretta si ottiene la seguente relazione:
	\[2\cdot bufcount=2\cdot bufcap= tmpsize = 2\cdot spf\]
	In quanto all'interno di \texttt{\small read\_port} si utilizzano le configurazioni
	già ottenute tramite \texttt{\small create\_conf\_port}, è in questo 
	punto che bisogna effettuare la vera modifica al codice.
	
	La prima significativa modifica sta nell'inserimento della variabile 
	\texttt{\small bytes\_per\_sample} all'interno della struttura dati 
	\texttt{\small struct conf\_port}, in quanto già dalla porta passata come
	argomento è possibile ottenere l'informazione dei \texttt{\small bits\_per\_sample}
	tramite il metodo seguente:
	\begin{clang}
	pjmedia_audio_format_detail *afd = pjmedia_format_get_audio_format_detail(&port->info.fmt, 1);
	\end{clang} 
	Conseguentemente utilizzo il valore di default \texttt{\small BYTES\_PER\_SAMPLE}
	unicamente quando non si possa disporre di tali informazioni.
\item 	Continuando ora con l'analisi dei valori ottenuti, voglio maggiorare
	il valore di $bufcap$. Impongo quindi che debba valere il seguente 
	sistema:
	\[\begin{cases}
	bufcount \leq bufcap\\
	2\cdot bufcap = tmpsize = cpspf \cdot cpbps 
	\end{cases}\]
	Questo valore è ottenuto dalla macro \texttt{\small PJMEDIA\_AFD\_SPF}
	definita all'interno del file:
	\begin{center}
	\texttt{\small \PJA/pjmedia/include/pjmedia/format.h}
	\end{center}
	dalla quale si può ricavare la seguente formula:
	\[cpspf = \mu ptime \cdot clock \cdot chan 10^{-6} = ptime \cdot clock \cdot chan 10^{-3}\]
	come per altro confermato dall'altra definizione in \texttt{\small wav\_player.c}
	che non fa utilizzo della macro sopra citata. 
	Per quanto concerne il valore utilizzato per $bufcap$, si considerino
	le seguente formule ricavate dalla funzione \texttt{\small create\_conf\_port}:
\[pptime=\frac{cpspf}{cpcha}\frac{10^3}{cpclock}\qquad cptime = \frac{cspf}{ccha}\frac{10^3}{cclock}\]
	dove il prefisso $p$ indica i valori propri della porta, mentre $c$
	quelli della \textit{conference port}. Dalla definizione di \texttt{\small buff\_ptime}
	originaria proposta di seguito:
	\begin{clang}
	if (port_ptime > conf_ptime) {
	  buff_ptime = port_ptime;
	  if (port_ptime \% conf_ptime)
	     buff_ptime += conf_ptime;
	} else {
	  buff_ptime = conf_ptime;
	  if (port_ptime \% conf_ptime)
	     buff_ptime += conf_ptime;
	}
	\end{clang}
	posso ottenere la seguente maggiorazione:
	\[buff\_ptime < \max\Set{pptime,cptime}+\min\Set{pptime,cptime} = pptime + cptime\]
	Svolgendo quindi la definizione di $bufcap$ data sempre dalla funzione in
	questione:
	\[\begin{split}
	bufcap &= cpclock \cdot buff\_ptime \cdot 10^{-3}\\
	       &= cpclock\cdot \left[10^3\left(\frac{cpspf}{cpcha\cdot cpclock} + \frac{cspf}{ccha\cdot cclock}  \right)\right]\cdot 10^{-3}\\
	       &= \left(\frac{cpspf}{cpcha} + \frac{cspf\cdot cpclock}{ccha\cdot cclock} \right)
	\end{split}\]
	
	Considerando quindi l'ulteriore aggiustamento fornito dal codice:
	\begin{clang}
	if (conf_port->channel_count > conf->channel_count)
	   conf_port->rx_buf_cap *= conf_port->channel_count;
	else
	   conf_port->rx_buf_cap *= conf->channel_count;
	\end{clang}
	e detto $\frac{1}{CRATE}=\frac{cpclock}{cclock} $ e supponendo che
	2 sia il numero massimo dei canali otteniamo:
	\[bufcap = \begin{cases}
	  cpspf + cpspf\frac{1}{CRATE} & cpcha > ccha\\
	  2(cpspf + cpspf\frac{1}{CRATE}) & cpcha \leq ccha
	\end{cases}\]
	Maggiorando quindi $cpspf cpspf\frac{1}{CRATE}$ con $2\cdot cpspf$,
	posso ulteriormente maggiorare $bufcap$ con:
	\[bufcap\leq 4\cdot cpspf\]
	da cui la mia correzione in:
	\begin{clang}
conf_port->rx_buf_cap = 2 * conf_port->samples_per_frame * dbld;
	\end{clang}
	Lascio tuttavia anche la correzione dei canali originaria, allo scopo di
	effettuare un'ulteriore maggiorazione nel caso in cui il numero di canali
	sia maggiore di 2.
\item Con i risultati ottenuti di cui sopra, che hanno consentito la modifica
	 del file proposto in Sottosezione \vref{subsec:conferencec}, si è
	 verificata la corretta riproduzione nel canale del file ottenuto dalla
	 risorsa esterna.
\end{enumerate}
