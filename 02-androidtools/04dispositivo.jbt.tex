\chapter{Preparazione dei dispositivi Android}
\minitoc\mtcskip

\section{Premesse: riconoscimento del dispositivo Android all'interno dell'ambiente GNU/Linux}
Per quanto riguarda il riconoscimento del dispositivo Android all'interno di un
sistema operativo GNU/Linux, ed in particolare per il riconoscimento dello stesso
da parte dell'SDK Android, è necessario eseguire il seguente script.
\lstinputlisting[language={bash}]{srcs/tablet_installer.sh}
In particolare, le operazioni di modifica di \texttt{\small fuse.conf} e l'aggiunta
dell'utente all'interno del sistema, sono necessarie unicamente alla prima
esecuzione dello stesso. Tramite questa operazione è già possibile interagire col 
dispositivo tramite i tool di Android come descritto nella Sezione \vref{sec:intdevandroid}.

In questo modo tuttavia non è presente l'applicazione \texttt{su} (o comunque essa 
non è utilizzabile), né è presente l'applicazione apk di Superuser per gestire
i permessi di superutente nelle applicazioni che si basano su Java.


\section{Rooting del dispositivo}\label{sec:rootingdevice}
Premessa:
se si intende compilare una versione dell'Android Kernel è opportuno, 
prima di effettuare il \textit{rooting} sul dispositivo, ottenere 
 il file \texttt{\small /proc/config.gz} tramite \texttt{\small adb}.
\bigskip

Il rooting del dispositivo si rivela necessario qualora si voglia rendere possibile
l'accesso ai privilegi di superutente ad un'applicazione: questo ad esempio è 
necessario per quelle applicazioni native che devono poter accedere a servizi forniti
dal sistema operativo che richiederebbero la descrizione dei permessi all'interno di un Manifest
File, per i motivi già illustrati nella Sezione \vref{sec:rivsupuser}. 
Questo è inoltre provato dal tentativo di esecuzione dell'applicazione
\texttt{\small pjsua} all'interno di una shell dove non sono stati richiesti i 
permessi di superutente; l'applicazione termina con l'errore che segue:
\begin{bash}
23:34:31.991    pjsua_aud.c  ....Opening sound device PCM@16000/1/20ms
23:34:31.992   opensl_dev.c  .....Creating OpenSL stream
23:34:31.996   opensl_dev.c  .....Recording stream type 4, SDK : 16
23:34:31.999   opensl_dev.c  .....Cannot realize recorder : 9
23:34:31.999   opensl_dev.c  .....Stopping stream
23:34:31.999   opensl_dev.c  .....OpenSL stream stopped
23:34:32.000   opensl_dev.c  .....OpenSL stream destroyed
\end{bash}
In particolare il numero d'errore 9 corrisponde al valore di SLResult \\
\texttt{\small SL\_RESULT\_CONTENT\_UNSUPPORTED}.
Tuttavia questo output è poco informativo se non si considera anche quello
fornito dal LogCat durante l'esecuzione dell'applicazione in questione:
\begin{bash}
W/ServiceManager(  106): Permission failure: android.permission.RECORD_AUDIO from uid=10043 pid=2978
E/AudioFlinger(  106): Request requires android.permission.RECORD_AUDIO
E/AudioRecord( 2978): AudioFlinger could not create record track, status: -1
E/libOpenSLES( 2978): android_audioRecorder_realize(0x2212d0) error creating AudioRecord object
W/libOpenSLES( 2978): Leaving Object::Realize (SL_RESULT_CONTENT_UNSUPPORTED)
\end{bash}
Possiamo infatti notare che il Service Manager non rende fruibile all'applicazione
l'\texttt{\small AndroidRecord}, in quanto l'applicazione eseguita in modalità utente
non detiene il permesso \texttt{\small android.permission.RECORD\_AUDIO} richiesto,
che tuttavia risulta essere inglobato nei permessi del superutente, in quanto
viene permessa l'esecuzione di tale operazione.
 
Posso inoltre notare come Android consenta che la connessione ad Internet venga
effettuata tramite le funzioni primitive della \textit{libc}, anche se si ritiene che
Google abbia implementato un meccanismo di Networking ``paranoico'' (\parencite[vedi][37]{libro:embedded})
che consenta solamente a certi processi, con peculiari \textit{capability}, di
accedere a tali servizi. Ricordo inoltre che, all'interno di Applicazioni Java
o di Applicazioni Native con JNI, è necessario specificare all'interno del manifest
l'intenzione di accedere ai servizi di rete.

Fornisco quindi di seguito alcune procedure di Rooting che sono state effettuate
allo scopo di valutare il porting di Pjproject.

\subsection{Rooting dell'emulatore}\label{subsec:rootingemu}
Per poter modificare la cartella \texttt{\small /system} che è montata in modalità di sola lettura, è necessario effettuarne il remounting nel modo seguente:
\begin{bash}
  ... mount -o rw,remount -t yaffs2 /dev/block/mtdblock0 /system
\end{bash}
Per installare l'applicazione, è sufficiente copiare il binario \texttt{\small su} (del quale sono presenti differenti versioni in rete) all'interno di \texttt{\small /system/xbin} tramite \texttt{\small adb}, ed installare l'apk \textit{Superuser} tramite il comando \texttt{\small install} di \texttt{\small adb}.

Alla fine della procedura possiamo rimontare \texttt{\small /system} come \textit{read-only} (ro); tuttavia possiamo mantenere abilitata la scrittura per poter effettuare delle ulteriori modifiche fino al prossimo riavvio, quando tale cartella verrà rimontata con soli permessi di lettura.

\subsection{Rooting di Samsung Galaxy Nexus}\label{subsec:rootingn}
 Al contrario dell'emultatore, i dispositivi Android hanno in genere il bootloader bloccato (\textit{locked}), che è sempre necessario sbloccare anche se, in questo modo, si infrange la garanzia. Allo scopo di effettuare l'\textit{unlocking}, è necessario entrare con il dispositivo nella modalità \textit{bootloader}. Per accedere in questa modalità o si invoca il comando \texttt{\small adb reboot bootloader} o, nel caso particolare del dispositivo Galaxy Nexus, spegnere il dispositivo, prima premere i tasti di aumento e diminuzione del volume contemporaneamente, e poi tenere premuto anche il tasto di accensione, finché non sarà visibile il bootloader. Il comando per l'\textit{unlocking} è il seguente:
\begin{bash} 
fastboot oem unlock
\end{bash}

 È sempre possibile ripristinare il bloccaggio (locking) in un secondo momento tramite il comando:
\begin{bash}
fastboot oem unlock
\end{bash}

 Bisogna ricordare che, con il primo comando di cui sopra, si compromettono i dati contenuti all'intero di \texttt{\small /sdcard}, le configurazioni dell'utente ed i programmi installati. Per ovviare a questo problema, è sempre possibile eseguire il seguente comando, allo scopo di effettuare un backup del dispositivo:
\begin{bash}
adb backup [-f <file>] [-apk|-noapk] [-shared|-noshared] [-all] [-system|-nosystem] [<packages...>]
\end{bash}

 È sempre possibile ripristinare tale backup in qualunque momento tramite l'altro comando che segue:
\begin{bash} 
adb restore <file>
\end{bash}

 Con questo comando si mantengono tutte le eventuali procedure di rooting apportate, ma non si recupereranno le modifiche effettuate sul file system del dispositivo, come la creazione di nuove cartelle o l'aggiunta di nuovi binari.
 A questo punto sarà possibile caricare un'immagine per poter effettuare in ogni momento la recovery; adoperando ad esempio Recovery Clockwork Touch, si può invocare il comando : 
\begin{bash}
fastboot flash recovery recovery-clockwork-touch-x.y.z.t-maguro.img
\end{bash}
 dove x.y.z.t è la versione a disposizione dell'immagine in questione. È possibile comunque ottenere le immagini accedendo all'indirizzo {\small \url{http://www.clockworkmod.com/rommanager}}. 
 
 In seguito è sufficiente fornire il file zippato per il rooting, che dovrà essere inserito all'interno della \texttt{\small /sdcard}, in modo da poterlo fornire alla \texttt{recovery-utility}. Sarà sempre possibile inoltre recuperare l'immagine originaria della recovery tramite l'indirizzo:
\begin{center}
{\small \url{https://dl.google.com/dl/android/aosp/yakju-jro03c-factory-3174c1e5.tgz}}
\end{center}
 È quindi sempre possibile ripristinare il dispositivo allo ``stato di fabbrica'' tramite il pacchetto fornito come sopra.
 
\subsection{Rooting di Olivetti Olipad}\label{subsec:rootingoli}
La procedura di Rooting dell'Olivetti Olipad è non standard, in quanto consente immediatamente di vedere quali sono le modifiche da effettuare al Ramdisk per consentire la modifica del filesystem \texttt{\small /system/xbin}; altro fattore non standard è l'utilizzo del tool \texttt{nvflash}, il quale è utilizzato per effettuare il flashing di dispositivi con processore Tegra 2.

In quanto questa procedura non è di particolare interesse ai fini dell'illustrazione dei tool messi a disposizione da Android, il codice sorgente dello script è messo a disposizione nella Sezione \vref{sec:rootolipad_src}.


