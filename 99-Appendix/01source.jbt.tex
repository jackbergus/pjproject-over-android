
\chapter{Tool SDK ed NDK}
\minitoc\mtcskip


\section{Script di interazione con l'SDK}\label{sec:scriintSDK}
\begin{bash}
#!/bin/bash
# Giacomo Bergami: questo file accede in modo uniforme ai programmi/script per
# la gestione dell'emultatore, del server e dei dispositivi
NDK=`pwd`/android-ndk-r8b
SDK=`pwd`/android-sdk-linux

echo "Using for NDK ${NDK}"
echo "Using for SDK ${SDK}"

if [[ $1 == "list-devices" ]]; then
	$SDK/tools/android list avds
elif [[ $1 == "new-device" ]]; then
	$SDK/tools/mksdcard $2 $3; 
	$SDK/tools/android create avd -n $2 -t $5 -sdcard $4
elif [[ $1 == "run-device" ]]; then
	$SDK/tools/emulator -avd $2 -partition-size 2047
elif [[ $1 == "running-devices" ]]; then
	$SDK/platform-tools/adb devices
elif [[ $1 == "linux-ports" ]]; then
	lsof -i :$2
elif [[ $1 == "linux-procs" ]]; then
	lsof +p $2
elif [[ $1 == "adb" ]]; then
	$SDK/platform-tools/adb ${@:2}
elif [[ $1 == "dadb" ]]; then
	$SDK/platform-tools/adb -s `./getdev` ${@:2}
elif [[ $1 == "dcpu" ]]; then
	$SDK/platform-tools/adb -s `./getdev` ${@:2} shell cat /proc/cpuinfo
elif [[ $1 == "android" ]]; then
	$SDK/tools/android ${@:2}
elif [[ $1 == "ndk-build" ]]; then
	$SDK/tools/android update project -p . -s
	$NDK/ndk-build
	ant debug 
else
	echo "$0 list-devices                                ottiene la lista degli emulatori creati"
	echo "$0 new-device name cardsize cardpos kernelver  crea un nuovo dispositivo"
	echo "$0 run-device name                             esegue un nuovo dispositivo"
	echo "$0 running-devices                             ottiene la lista dei dispositivi in esecuzione"
	echo "$0 linux-ports port	                     ottiene la lista dei processi sulla data porta"
	echo "$0 linux-procs pid                             ottiene i fd di un processo"
	echo "$0 adb ...                                     per gli altri comandi di interazione col dispositivo"
fi;
\end{bash}

\section{Tentativi di configurazione per la crosscompilazione}\label{sec:tentconfcross}
Per una descrizione dettagliata delle problematiche riscontrate, faccio riferimento
alla Sezione \vref{sec:conscrossocmp}.

\subsection{Primi passi ed introduzione dello script \texttt{\small myConf}}
Effettuo la definizione dello script \texttt{\small myConf} per l'\textit{export} delle variabili
necessarie allo script \texttt{configure-android}, e più preesistente:
\begin{bash}
#!/bin/bash
export ANDROID_NDK=/home/jack/android-ndk-r8b
export API_LEVEL=5
export PATH=$PATH:$ANDROID_NDK

#cp -av /usr/share/misc/config.guess ./
#cp -av /usr/share/misc/config.sub  ./

autoconf aconfigure.ac > aconfigure
chmod +x aconfigure
chmod +x configure

echo "#define PJ_CONFIG_ANDROID 1
#include <pj/config_site_sample.h>" > pjlib/include/pj/config_site.h

./configure-android --disable-floating-point --disable-large-filter $* 
\end{bash}


In particolare lo script \texttt{configure-android} è stato modificato alla fine nel modo seguente:
\begin{bash}
export LDFLAGS=" -nostdlib -L${ANDROID_SYSROOT}/usr/lib/"
export LIBS=" -lc -lgcc -lm"
export CFLAGS=" -Wl,-Bstatic -lm -I${ANDROID_SYSROOT}/usr/include"
export CPPFLAGS="${CFLAGS}"
export CXXFLAGS=" -Wl,-Bstatic -lm --sysroot=${ANDROID_SYSROOT}"

./configure --host=${TARGET_HOST} --disable-video
\end{bash}


In particolare si è continuato ad utilizzare il \textit{toolchain} \texttt{\small arm-linux-androideabi-4.4.3}.
In questo caso si sono presentati i seguenti problemi:
\begin{itemize}
\diam Il linker non è in grado di trovare il simbolo \texttt{\_start} ed applica
	conseguentemente il defaulting: la risoluzione di questo problema è stata 
	già illustrata nella Sezione \vref{sec:assstart}, assieme alla valutazione
	dell'errore  \textit{Unknown EABI object attribute 44}.
\diam Effettuando il linking statico non si trova la libreria OpenSLES: aggiungo
	quindi il flag \texttt{\small -lOpenSLES} e cambio il numero delle api alla
	versione 14. 
\end{itemize}

\subsection{Permanenza dell'errore di Segmentation Fault nell'esecuzione}
Per cercare di rimediare all'errore di Segmentation Fault, ho tentato di 
apportare le seguenti modifiche:
\begin{itemize}
\item Si continua ad ottenere \texttt{\small Segmentation Fault} senza aver specificato
	i flag \texttt{\small --disable-floating-point --disable-large-filter} in fase di 
	configurazione, dopo aver sostituito \texttt{\small -Wl,-Bstatic} con
	\texttt{shared} e senza aver ancora linkato il binario \texttt{crt0.o}
\item Anche dopo l'aggiunta dei flag \texttt{\small --disable-floating-point --disable-large-filter},
	si ha l'interruzione dell'applicazione con \texttt{\small SIGFAULT}: l'errore
	è quindi indipendente dalla gestione numerica.
\item Cambiando la configurazione Android in questo modo:
\begin{bash}
export LDFLAGS=" -nostdlib -L${ANDROID_SYSROOT}/usr/lib/"
export LIBS=" -lc -lgcc -lm "
export CFLAGS=" -Wl,-Bstatic  -lm -I${ANDROID_SYSROOT}/usr/include"
export CPPFLAGS="${CFLAGS}"
export CXXFLAGS=" -Wl,-Bstatic -lm --sysroot=${ANDROID_SYSROOT}"
\end{bash}
       sono costretto ad omettere il flag \texttt{\small -lOpenSLES}, in quanto
       altrimenti la compilazione viene terminata con insuccesso nella fase
       iniziale di escuzione dello script di configurazione. 
\end{itemize}

\subsection{Stadio intermedio di configurazione}
Fornisco di seguito gli script di configurazione che portano alla corretta
esecuzione di \texttt{pjtest}, ma non all'esecuzione di \texttt{pjsua} per l'errata 
lettura degli argomenti come parametro: come si può riscontrare dalla configurazione,
ciò era dovuto al fatto che importavo il file \texttt{crt0.o} solamente all'interno
di \texttt{\small PJ\_LDLIBS}. Queste configurazioni inoltre fanno ancora riferimento
al trunk di \textit{pjproject}, e non al \textit{branch} per Android.

\lstinputlisting[language=bash,caption= user.mak]{srcs/user_intermediate.mak}

Come possiamo notare da questo script \texttt{\small myConf}, sono appunto costretto
a disabilitare alcuni supporti per i codec audio in quanto questi non erano
supportati dalla versione \textit{trunk}.

\lstinputlisting[language=bash,caption= myConf]{srcs/myConf_intermediate}

Riporto inoltre lo script \texttt{\small configure-android} di seguito:

\lstinputlisting[language=bash,caption= configure-android,label=alg:confandroid]{srcs/configure-android_intermediate}

\section{Sorgente C del programma Client/Server d'esempio}\label{sec:cliserv}
\lstinputlisting[language=C,caption= Client/Server d'esempio]{srcs/cliserver.c}

\section{NDK ed Assembly per la definizione di \texttt{\_start}}\label{sec:assstart}
In seguito si fornisce la definizione dell'entry point \texttt{\_start}, che è 
possibile ottenere dal disassembling dei binari forniti da Google.
\lstinputlisting[language={[x86masm]Assembler}]{srcs/crt0.s}

Si è scoperto, in seguito alla scrittura di questo sorgente, che esso era anche
contenuto all'interno degli oggetti \texttt{\small crtbegin\_static.o} e 
\texttt{\small crtbegin\_dynamic.o}. Questo è stato però notato solamente dopo
aver letto i sorgenti Android di \textsc{Bionic}. Questi oggetti tuttavia non
sono ancora contenuti all'interno della libc, né viene precisato all'interno
del documento \texttt{\small CHANGES.html} che sia effettivamente necessario linkare questi oggetti 
per ottenere l'esecuzione di \texttt{\small main}.
\begin{quotation}
\textit{
Fixed \_\_start (in crtbegin\_dynamic/static.o) to "call \_\_libc\_init" instead of
  "jump \_\_libc\_init", otherwise stack unwinding past \_\_libc\_init may get wrong return
  address and crash the program or do wield things.  With "call", return address is pushed
  on stack and unwinding stops correctly at \_start.  Note that \_\_libc\_init never returns, so
  this fix won't affect normal program execution.  But just in case it "does" return,
  jump to address 0 and halt.
}
\end{quotation}
Tuttavia bisogna notare che il riconoscimento della funzione \texttt{\small \_start}
è effettuata unicamente nella versione 4.6 della versione del tool di crosscompilazione
dell'NDK, all'interno della quale sono però presenti dei 
\texttt{\small crtbegin\_dynamic.o} che, se inclusi nella versione 4.6, portano al
seguente errore:
\begin{bash}
Unknown EABI object attribute 44
\end{bash}
Questo in genere avviene quando le versioni dello stesso cross-compilatore non
coincidono: per questo si è reso necessario riutilizzare l'oggetto definito sopra.

Si è inoltre notato, tramite la non corretta modifica dello script di crosscompilazione
di Pjproject, che una non corretta inclusione di questo binario, oltre a lasciare
al crosscompilatore la decisione di quale funzione utilizzare in luogo della
funzione \texttt{\small \_main}, non consente di inzializzare correttamente gli
argomenti da passare alla funzione \texttt{\small main}: in particolare il valore di
\texttt{\small argc} non coinciderà con l'effettivo numero di parametri passati, ed
inoltre \texttt{\small argv} conterrà array di tutti puntatori a \texttt{\small NULL}.

\section{Rooting di Olivetti Olipad}\label{sec:rootolipad_src}
Questo sorgente è tratto dal Tool \texttt{Medion LifeTab P9514-Root Tool}. Lo script
è stato tradotto dal Tedesco in Inglese.
\lstinputlisting[language=bash]{srcs/root-olitab.sh}

Qui di seguito mostro invece le opzioni che possono essere utilizzate al fine
di interagire con il tool \texttt{nvflash}:
\begin{bash}
Nvflash started
nvflash action [options]
action (one or more) =
   --help (or -h)
       displays this page
   --cmdhelp cmd(or -ch)
       displays command help 
   --resume (or -r)
       send the following commands to an already-running bootloader
   --quiet (or -q)
       surpress excessive console output
   --wait (or -w)
       waits for a device connection (currently a USB cable)
   --create
       full initialization of the target device using the config file
   --download N filename
       download partition filename to N
   --setboot N
       sets the boot partition to partition N
   --format_partition N
       formats contents of partition N
   --read N filename
       reads back partition N into filename
   --getpartitiontable filename
       reads back the partition table into filename
   --getbit filename
       reads back BIT into filename
   --getbct
       reads back the BCT from mass storage
   --odm C Data
        ODM custom 32bit command 'C' with associated 32bit data
   --go
       continues normal execution of the downloaded bootloader
    options =
   --configfile filename
       indicates the configuration file used with the following commands:
       --create, --format_all
   --bct filename
       indicates the file containing the BCT
   --sbk 0x00000000 00000000 00000000 00000000
       indicates the secure boot key for the target device
   --bl filename
       downloads and runs the bootloader specified by filename
   --odmdata N
      sets 32bit customer data into a field in the BCT, either hex or
      decimal
   --diskimgopt N
       sets 32bit data required for disk image convertion tool
   --format_all
       formats all existing partitions on the target device using the config file,
       including partitions and the bct
   --setbootdevtype S
       sets the boot device type fuse value for the device name.
       allowed device name string mentioned below:
           emmc, nand_x8, nand_x16, nor, spi
   --setbootdevconfig N
       sets the boot device config fuse value either hex or decimal
   --verifypart N
        verifies data for partition id = N specified. N=-1 
        indicates all partitions
        Intended to be used with --create command only.
   --setbct
       updates the chip specific settings of the BCT in mass storage to
       the bct supplied,used with --create, should not be with --read,and
       --format(delete)_all,format(delete)_partition,--download, and--read
   --sync
       issues force sync commad
   --rawdeviceread S N filename
       reads back N sectors starting from sector S into filename
   --rawdevicewrite S N filename
       writes back N sectors from filename to device starting from sector S 
\end{bash}




