\chapter{Pjproject - modifiche al codice}
\minitoc\mtcskip

Per poter effettuare la compilazione della libreria e delle applicazioni in 
ambiente Android tramite la versione \textit{trunk} di \textit{Pjproject}, è stato 
necessario effettuare una patch al sorgente.

Si consiglia di
eseguire prima il comando:
\begin{center}
\texttt{\small patch -{}-dry-run -p1 -i PATCH}
\end{center}
per verificare l'effettiva conformità della patch al sorgente senza applicarla, 
e quindi ripetere lo stesso comando omettendo il flag \texttt{\small -{}-dry-run}.
Le modifiche minori attinenti al processo di configurazione del crosscompilatore
e dell'aggiunta degli oggetti binari è stata riportata nella Sezione \vref{sec:tentconfcross}.



\section{Gestione dei file audio}
\subsection{Lettura degli header del file WAVE}\label{subsec:appreadwave}
Qui di seguito riporto il codice che ho effettuato per la lettura dei primi header
dei file WAVE.

\lstinputlisting[language=C, caption=Lettura header WAVE]{srcs/wave_c_sample.c}

\subsection{\texttt{\small conference.c}}\label{subsec:conferencec}
Oltre alle altre modifiche minori che ho già avuto modo di illustrare, riporto
qui di seguito il codice completo delle funzioni che sono state da me modificate
all'interno del file \texttt{\small conference.c}. Le parti da me modificate sono
marcate da \texttt{\small jb09}.

\lstinputlisting[language=C, caption= Modifiche a conference.c]{srcs/new_conference.c}
